\documentclass{morelull}

\title{morelull使用说明}
\author{荀徒之}
\date{\today}

\begin{document}

\maketitle
\tableofcontents

\section{小特色}
待续。

\section{中文标点符号}

\begin{itemize}
\item{专名号:\专名{荀徒之}。}
\item{着重号:早上\着重{日}出\着重{东方}。}
\item{内联代码:庄重地写下了\代码{hello world}。}
\end{itemize}

\section{中看不中用的提示文本框}

\subsection{提示作用}
\begin{提示}
    \begin{itemize}
    \item{多边形是由同一平面内若干条不在同一直线上的线段组成;}
    \item{是平面内的一些线段首尾顺次相连形成的封闭图形;}
    \item{多边形的顶点数、边数、及角的个数相等;}
    \item{.多边形对角线的条数:n(n-3)/2;}
    \end{itemize}
\end{提示}

\subsection{提醒、警告作用}
\begin{提醒}
    三角形的角平分线和这个角对边相交,这个顶点和交点的线段叫角平分线。

    如果一个直角三角形的斜边和一条直角边与另一个直角三角形的斜边和一条直角边对应成比例,那么这两个直角三角形相似。
\end{提醒}

\subsection{注意、强调作用}
\begin{注意}
除了我们要用一般方法判定三角形相似之外,还有一些相似模型分享给大家。当我们看到这类模型的时候,就会直观的得出里面存在相似三角形(当然,解答题还是要写推理过程的),从而提高我们做题速度。简直是提速神器。
\end{注意}

\section{扩展计算机语言高亮}

listings支持语法并不广泛,为了满足更多需求,我定义新的两种语法。

\subsection{javascript(js)语法高亮}
\lstinputlisting[language=js]{./example.js}

\subsection{rust(rs)语法高亮}
\lstinputlisting[language=rs]{./example.rs}

\end{document}